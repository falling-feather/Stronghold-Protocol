% \documentclass{article} % article 文档
% \usepackage[UTF8]{ctex}  % 使用宏包(为了能够显示汉字)
% % 设置页面的环境,a4纸张大小,左右上下边距信息
% \usepackage[a4paper,left=10mm,right=10mm,top=15mm,bottom=15mm]{geometry}

% \title{[归元]项目报告}  % 文章标题
% \author{落入白川的羽}   % 作者的名称
% \date{\today}       % 当天日期

% % 正文开始
% \begin{document}

% \maketitle          % 添加这一句才能够显示标题等信息

% % 生成目录设置
% \renewcommand{\contentsname}{目录} %将content转为目录
% \tableofcontents

% % 摘要开始部分
% \begin{abstract}
% 该部分内容是放置摘要信息的。该部分内容是放置摘要信息的。该部分内容是放置摘要信息的。该部分内容是放置摘要信息的。该部分内容是放置摘要信息的。
% \end{abstract}

% % 标题开始
% \section{一级标题1}
% 第一段一级标题下的内容,一级标题下的内容,一级标题下的内容,一级标题下的内容,一级标题下的内容,一级标题下的内容,一级标题下的内容,一级标题下的内容。\par
% 第二段一级标题下的内容,一级标题下的内容,一级标题下的内容,一级标题下的内容,一级标题下的内容,一级标题下的内容,一级标题下的内容,一级标题下的内容。

% \subsection{二级标题1.1}
% 二级标题下的内容。

% \subsubsection{三级标题下的内容1.1.1}
% 三级标题下的内容。

% \section{一级标题2}
% 一级标题2中的内容

% % 正文结束
% \end{document}
\documentclass{beamer}

% 下面的内容会在标题页上展示
\title{Hello, World!}
\author{Studying Father}
\institute{SFOI Team}
\date{September 12th, 2019}

\begin{document}% 下面都是要显示的内容
	\frame{\titlepage}% 显示标题页
	
	\begin{frame}
		\frametitle{Hello, World!}
		Hello, World!
	\end{frame}
    
\end{document}
